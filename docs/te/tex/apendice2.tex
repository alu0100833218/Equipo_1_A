\section{Explicacion del codigo \textsf{Python} }
\label{Apendice2:label}

Como dijimos en el apendice anterior, se trata del codigo de un programa realizado en python para calcular la biseccion de f(x)=Cos($\pi$x)  en un intervalo introducido por el usuario junto con el margen de error. ademas calcula el tiempo de ejecucion y dibuja la grafica en otro intervalo que desee el usuario.\par

Primero importamos las librerias necesarias. En este caso las necesitamos para calcular el tiempo, para las funciones trigonometricas (Coseno), y para representar graficas.

\begin{verbatim}
#! /usr/bin/python
#!encoding: UTF-8
import time
import math
import matplotlib.pyplot as pl
import numpy as np
import sys
\end{verbatim}

Definimos las dos funciones que vamos a usar principalmente, una para calcular los valores de f(x) al sustituir un x en ella, y otra para calcular la biseccion.

\begin{verbatim}
def f(x):
  return (math.cos(x*math.pi))
  
def biseccion(a,b,e):
  c=(a+b)/2.0
  while((f(c)!=0.00000001) and (abs(b-a)>e)):
    if f(a)*f(c)<0.00000001:
      b=c
    else:
      a=c
    c=(a+b)/2.0
  return c
\end{verbatim}

A continuacion pedimos al usuario que ingrese los extremos superior e inferior del intervalo y la tolerancia o margen de error.
Establecemos una condicion para poder realizar la biseccion, y es que exista un cambio de signo entre f(A) y f(B), asi que f(A)*f(B)<0. Cuando esto se cumple el programa se ejecuta, empezando a contar el tiempo y parando este "cronometro" despues del calculo.


\begin{verbatim}
A=float(raw_input("Introduzca el extremo inferior del intervalo (a) en el que se desea buscar la raiz: "))
B=float(raw_input("Introduzca el extremo superior del intervalo (b) en el que se desea buscar la raiz: "))
E=float(raw_input("Introduzca el margen de error a partir del cual no afecte demasiado a sus calculos: "))
if f(A)*f(B)<0.00000001:
  start=time.time()
  r=biseccion(A,B,E)
  finish=time.time()-start
  print "La raiz que se ha calculado en ese intervalo de forma aproximada es:%4.3f" %r
  print "El tiempo que ha tardado en ejecutarse el calculo en segundos es:"
  print finish
  
else:
  print "No podemos asegurar que en el intervalo introducido exista raiz, lo sentimos."
\end{verbatim}


Tras hallar la biseccion deseada, se pide al usuario un valor de x para representar la funcion en un intervalo de -x a x. Con la ayuda de un bucle for conseguimos dar valores desde el 0 hasta x y guardarlos en un vector en los valores de y. De este modo podemos tener los puntos a representar.

\begin{verbatim}
x=int(raw_input("Introduzca el maximo valor de x para la representacion: "))
lista=[]
for i in range(x):
  y=math.cos(math.pi*i)
  lista.append(y)
\end{verbatim}

Por ultimo representamos la funcion. Elejimos el color y el ancho de la linea. Como nos interesa ver el corte con el eje de abcisas, representamos una recta x multiplicada por 0, asi todos sus puntos valdran 0. Le ponemos un titulo a la grafica, la guardamos y la mostramos. 
\begin{verbatim}
pl.figure(figsize=(8,6), dpi=80)

X = np.linspace(-x, x, 256, endpoint=True)
C = np.cos(X*np.pi)
S = 0*(X)

pl.plot(X,C, color="cyan", linewidth=2.5, linestyle="-", label="Coseno")
pl.plot(X,S, color="black", linewidth=1.5, linestyle="-", label="Eje X")
pl.legend(loc='upper left')
pl.xlim(X.min()*1.1,X.max()*1.1)
pl.ylim(C.min()*1.1,C.max()*1.1)
pl.yticks([-1, 0, +1])
pl.title("Representacion grafica")
pl.savefig("cos.eps", dpi=72)
\end{verbatim}

Y aqui termina el codigo.