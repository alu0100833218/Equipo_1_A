%%%%%%%%%%%%%%%%%%%%%%%%%%%%%%%%%%%%%%%%%%%%%%%%%%%%%%%%%%%%%%%%%%%%%%%%%%%%%
% Chapter 1: Motivaci�n y Objetivos 
%%%%%%%%%%%%%%%%%%%%%%%%%%%%%%%%%%%%%%%%%%%%%%%%%%%%%%%%%%%%%%%%%%%%%%%%%%%%%%%
Este m�todo , que se utiliza para resolver ecuaciones de una variable, est� basado en el "Teorema de los Valores Intermedios" (TVM), en el cual se establece que toda funci�n continua 'f', en un intervalo cerrado $[a,b]$, toma todos los valores que se hallan entre $f(a)$ y $f(b)$, de tal forma que la ecuaci�n $f(x)=0$ tiene una sola ra�z que verifica $f(a)*f(b)<0$.

%---------------------------------------------------------------------------------
\section{�Qu� es el m�todo de bisecci�n?}
\label{1:sec:1}

  El m�todo de la bisecci�n es un algoritmo que nos permite encontrar una ra�z de la funci�n que tengamos. Su funcionamiento es muy similar al de la b�squeda binaria, s�lo que estamos utilizando valores continuos ($\infty$) en lugar de discretos (finitos). Vamos a asumir que las funciones con las que trabajamos son continuas.

  Sabemos que si la funci�n $f(x)$ atraviesa el eje x, en la intersecci�n existe una ra�z, por lo que antes de la ra�z $f(x) > 0$ y despu�s $f(x) < 0$. Entonces, teniendo dos puntos 'a' y 'b' tales que $f(a)$ y $f(b)$ tienen signos opuestos, sabemos que debe haber una ra�z en el intervalo $[a, b]$.

%---------------------------------------------------------------------------------
\section{�C�mo funciona su algoritmo?}
\label{1:sec:2}
  El algoritmo funciona de la siguiente forma: 

\begin{itemize}
  \item Primero tomamos el punto medio entre 'a' y 'b', al cual llamaremos $c ()$. Si el valor de $f(c)=0$, o est� suficientemente cerca.
  \item Segundo tomamos a 'c' como el valor de la ra�z.
\end{itemize}

  En caso contrario:
  
\begin{itemize}
  \item Primero reemplazamos a 'a' o 'b' por 'c', de acuerdo al signo de $f(c)$, de tal forma que los signos de $f(a)$ y $f(b)$ sigan siendo diferentes.
\end{itemize}
  
  Finalmente, repetimos el m�todo con el nuevo intervalo.

%---------------------------------------------------------------------------------
\section{Ventaja del m�todo de bisecci�n.}
\label{1:sec:3}
  
  La principal ventaja de este m�todo es que es muy eficaz aunque menos que el m�todo de Newton, ya que, est� garantizado que el m�todo converge si los valores de $f(a)$ y $f(b)$ son de signos contrarios. La convergencia de este m�todo es lineal, y su error absoluto despu�s de n iteraciones es: $\frac{|b-a|}{2^n}$