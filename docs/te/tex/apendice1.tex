\section{Codigo\_\textsf{Python}}
\label{Apendice1:XXX}

\begin{center}
\begin{footnotesize}
\begin{verbatim}
###################################################################################
# solucion .py
###################################################################################
#
# David Tomas Montesdeoca Flores y Carmen Laura Martin Gonzalez
#   
# 11-05-2014
#
# Es un programa que calcula la biseccion de la funcion f(x)=cos(pix) en un intervalo que 
# que debe introducir el usuario junto con un margen de error. Además de calcular 
# las raices de la funcion, calcula el tiempo de ejecucion y dibuja la grafica en 
# otro intervalo distinto pedido al usuario.
#
###################################################################################
#! /usr/bin/python
#!encoding: UTF-8
import time
import math
import matplotlib.pyplot as pl
import numpy as np
import numpy as np
import sys

def f(x):
  return (math.cos(x*math.pi))
  
def biseccion(a,b,e):
  c=(a+b)/2.0
  while((f(c)!=0.00000001) and (abs(b-a)>e)):
    if f(a)*f(c)<0.00000001:
      b=c
    else:
      a=c
    c=(a+b)/2.0
  return c


A=float(raw_input("Introduzca el extremo inferior del intervalo (a) en el que se desea buscar la raiz: "))
B=float(raw_input("Introduzca el extremo superior del intervalo (b) en el que se desea buscar la raiz: "))
E=float(raw_input("Introduzca el margen de error a partir del cual no afecte demasiado a sus calculos: "))
if f(A)*f(B)<0.00000001:
  start=time.time()
  r=biseccion(A,B,E)
  finish=time.time()-start
  print "La raiz que se ha calculado en ese intervalo de forma aproximada es:%4.3f" %r
  print "El tiempo que ha tardado en ejecutarse el calculo en segundos es:"
  print finish
  
else:
  print "En el intervalo introducido no existe raiz, lo sentimos."
  
x=int(raw_input("Introduzca el maximo valor de x para la representacion: "))
lista=[]
for i in range(x):
  y=math.cos(math.pi*i)
  lista.append(y)
  

pl.figure(figsize=(8,6), dpi=80)

X = np.linspace(-x, x, 256, endpoint=True)
C = np.cos(X*np.pi)
S = 0*(X)

pl.plot(X,C, color="cyan", linewidth=2.5, linestyle="-", label="Coseno")
pl.plot(X,S, color="black", linewidth=1.5, linestyle="-", label="Eje X")
pl.legend(loc='upper left')
pl.xlim(X.min()*1.1,X.max()*1.1)
pl.ylim(C.min()*1.1,C.max()*1.1)
pl.yticks([-1, 0, +1])
pl.title("Representacion grafica")
pl.savefig("cos.eps", dpi=72)
pl.show()
###################################################################################

\end{verbatim}
\end{footnotesize}
\end{center}

\section{Codigo\_\textsf{Python}\_para\_el\_Hardware}
\label{Apendice1:YYY}

\begin{center}
\begin{footnotesize}
\begin{verbatim}
/###################################################################################
 # prct12_1 .py
 ###################################################################################
 #
 # David Tomas Montesdeoca Flores y Carmen Laura Martin Gonzalez
 #
 # 11-05-2014
 #
 # Este programa nos muestra la versión de Linux que estamos utilizando y lo guarda 
 # en un fichero de texto.
 #
 ##################################################################################
 #!encoding: UTF-8
#!usr/bin/python

import os
import platform

def SOFTinfo():
    softinfo={}
    softinfo={'Several':platform.uname(), 'S.O':platform.platform(), 'Pythons Version':platform.python_version(), 'Date':platform.python_build()}
    return softinfo
    
def CPUinfo():
# infofile on Linux machines:
  infofile = '/proc/cpuinfo'
  cpuinfo = {}
  if os.path.isfile(infofile):
    f = open(infofile, 'r')
    for line in f:
      try:
        name, value = [w.strip() for w in line.split(':')]
      except:
        continue
      if name == 'model name':
        cpuinfo['CPU type'] = value
      elif name == 'cache size':
        cpuinfo['cache size'] = value
      elif name == 'cpu MHz':
        cpuinfo['CPU speed'] = value + ' Hz'
      elif name == 'vendor_id':
        cpuinfo['vendor ID'] = value
    f.close()
  return cpuinfo

if __name__ == '__main__':
  softinfo = SOFTinfo()
  for keys in softinfo.keys():
    print softinfo[keys]
  cpuinfo = CPUinfo()
  for keys in cpuinfo.keys():
    print cpuinfo[keys]
  
  print "Introduzca el  ##################################################################################nombre del fichero para almacenar los resultados: "
  nombre_fichero = raw_input();
  f = open(nombre_fichero, "w")
  for keys in softinfo.keys():
    if type (softinfo[keys]) is list:
      f.write('\n'.join(softinfo[keys]))
    else:
      f.write(str(softinfo[keys]))
      f.write('\n')
  for keys in cpuinfo.keys():
    if type(cpuinfo[keys]) is list:
      f.write('\n'.join(cpuinfo[keys]))
    else:
      f.write(str(cpuinfo[keys]))
      f.write('\n')
  f.close()

  
platform.uname()
platform.platform()
platform.python_version()
platform.python_build()
 ##################################################################################
 
\end{verbatim}
\end{footnotesize}
\end{center}
