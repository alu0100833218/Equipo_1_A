%%%%%%%%%%%%%%%%%%%%%%%%%%%%%%%%%%%%%%%%%%%%%%%%%%%%%%%%%%%%%%%%%%%%%%%%%%%%%
% Chapter 4: Conclusiones y Trabajos Futuros 
%%%%%%%%%%%%%%%%%%%%%%%%%%%%%%%%%%%%%%%%%%%%%%%%%%%%%%%%%%%%%%%%%%%%%%%%%%%%%

Para concluir tenemos que mencionar varios puntos del trabajo:

\begin{itemize}

\item \textsf{Python} esta en movimiento y en pleno desarrollo, pera ya es una realidad y una interesante opcion para realizar todo tipo de programas que se ejecuten en cualquier maquina. El equipo de desarrollo esta trabajando de manera cada vez mas organizada y cuentan con el apoyo de una comunidad que esta creciendo rapidamente.

\item Algunas de las ventajas que tiene \LaTeX{} son que, crea con facilidad estructuras complejas como pies de pagina, indices, tablas, etc, y para la rama de ciencias tiene principalmente una ventaja primordial que es que se pueden crear con facilidad formulas matematicas.

\item \textsc{Beamer} tiene todas las ventajas de \LaTeX{}. Su presentacion en PDF es estandar y portable. Ademas tiene unos estilos predefinidos con botones de navegacion, tablas de contenidos, pies de pagina informativos, etc.

\item El metodo de la Biseccion converge lentamente, lo que genera la propagacion de error por la cantidad de operaciones e iteraciones necesarias para que el metodo converja.

\end{itemize}

Finalmente, al complementar los elementos mencionados anteriormente y centrandonos en el calculo de las raices mediante el metodo de biseccion, hemos aprendido a crear un fichero de texto y una presentacion a modo de diapositivas, con un cierto grado de exigencia. Tambien hemos aprendido a ser independientes buscando la informacion correcta sobre el tema propuesto y a la vez aprender a trabajar en equipo.
Bajo nuestro punto de vista este es el objetivo de la programacion de este trabajo, ya que, esto nos serviria para los trabajos futuros que tengamos que realizar.