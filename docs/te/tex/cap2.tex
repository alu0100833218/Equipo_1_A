%%%%%%%%%%%%%%%%%%%%%%%%%%%%%%%%%%%%%%%%%%%%%%%%%%%%%%%%%%%%%%%%%%%%%%%%%%%%%%%
% Chapter 2: Fundamentos Te�ricos 
%%%%%%%%%%%%%%%%%%%%%%%%%%%%%%%%%%%%%%%%%%%%%%%%%%%%%%%%%%%%%%%%%%%%%%%%%%%%%%%

%++++++++++++++++++++++++++++++++++++++++++++++++++++++++++++++++++++++++++++++

Este trabajo trata sobre la bisecci�n de la funci�n $f(x)=cos(\pi*x)$. Para realizarlo utilizamos 
programas como \LaTeX{} para la realizacion del pdf, \textsc{Beamer} para realizar la presentaci�n y 
Python para calcular las raices de la funci�n nombrada anteriormente y crear su grafica correspondiente.

%++++++++++++++++++++++++++++++++++++++++++++++++++++++++++++++++++++++++++++++

\section{\LaTeX{}}
\label{2:sec:1}
  \LaTeX{} es un paquete de macros que permite a los autores componer e imprimir su trabajo con la mayor calidad tipogr�fica posible, usando un formato profesional predefinido. \LaTeX{} due escrito originalmente por Leslie Lamport. Emplea el formateador de \TeX{} como motor de composici�n.
  
  \TeX{} es un programa de ordenador creado por Donald E. Knuth. Sirve oara componer texto y f�rmulas matem�ticas. 


\section{\textsc{Beamer}}
\label{2:sec:2}
  \textsc{Beamer} fue desarrollado por Till Tantau para la presentaci�n de su tesis doctoral, como un conjunto de macros que facilitara el uso de otras clases, como seminar o prosper. 
  
  \textsc{Beamer} es una clase de \LaTeX que permite crear diapositivas. Aunque tambi�n puede ser utilizada para crear presentaciones dinamicas que pueden ser proyectadas desde el ordenador.
  
\section{Python}
\label{2:sec:3}
  Python es un lenguaje de programaci�n creado por Guido van Rossum a principios de los a�os 90 cuyo nombre est� inspirado en el grupo de c�micos ingleses "Monty Python". Es un lenguaje similar a Perl, pero con una sintaxis muy limpia y que favorece un c�digo legible.
  
  Se trata de un lenguaje interpretado o de script, con tipado din�mico, fuertemente tipado, multiplataforma y orientado a objetos
