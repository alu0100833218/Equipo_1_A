\documentclass{beamer}
\usepackage[spanish]{babel}
\usepackage[utf8]{inputenc}
\usepackage{graphicx}

\title[La bisección de $f(x)=cos($\pi$x)$ en \textsc{Beamer}]{Bisección con $f(x)=cos($\pi$x)$}
\author[D. Montesdeoca y  L. Martín]{Carmen Laura Martín González
y
David Tomás Montesdeoca Flores}
\date[11/05/14]{11 de mayo de 2014}
\usetheme{Madrid}


\begin{document}

\begin{frame}
\titlepage
\end{frame}

\begin{frame}
\frametitle{Indice}
\tableofcontents[pausesections]

\end{frame}

\section{Definición de Bisección}

\begin{frame}
\frametitle{Definición de Bisección}

Según la RAE, La bisección es la acción o efecto de bisecar,es decir, dividir a la mitad y se aplica generalmente en la división de ángulos. Aunque esta definición no se aleja mucho de la deseada, laque verdaderamente nos interesa es la siguiente:

\begin{block}{Definición aplicada}
El método de bisección es un algoritmo usado en matemáticas para llevar a cabo una búsqueda de raíces, y el cual se realiza dividiendo el intervalo a la mitad y seleccionando el subintervalo de estos que contiene la raíz. Cuantas más cifras decimales queramos obtener más divisiones tendremos que realizar.
\end{block} 

\end{frame}

\section{Historia del Cálculo del número $\pi$}

\begin{frame}
\frametitle{Historia del Cálculo del número $\pi$}
El cálculo del número $\pi$ a lo largo de la historia ha sido una ardua tarea para los cientificos que han llevado a cabo sus aproximaciones.

Algunas de sus aproximaciones a lo largo de la historia más importantes han tenido lugar en:

\begin{itemize}
  \item El Antiguo Egipto.
  \pause
  \item La Antigüedad Clásica (Grecia y Roma).
  \pause
  \item Mesopotamia.
  \pause
  \item La India.
  \pause
  \item China.
  \pause
  \item Europa.
  \pause
  \item Persia.
  \pause
\end{itemize}

En la época actual el mayor numero de decimales obtenido se llevó  a cabo por Shigeru Kondo, obteniendo 10.000.000.000.000 cifras.

\end{frame}

\begin{frame}
\frametitle{Historia del Cálculo del número $\pi$}
\begin{block}{Tabla de decimales obtenidos}
\begin{table}{}
\begin{tabular}{lrcl}
Año   &  Nombre       &  Ordenador &  Número de decimales \\ \hline
1949  &  Reitwiesner  &  ENIAC     &  2.037  \\ \hline
1959  &  Guilloud     &  IBM 704   &  16.167  \\ \hline
1986  &  Bailey       &  CRAY-2    &  29.360.111 \\ \hline
2011  &  Kondo        &            &  10.000.000.000.000 \\ \hline
\end{tabular}
\end {table}
\end{block}
\end{frame}

\section{Algunas fórmulas que contienen el número $\pi$}
\subsection{Geometría} 

\begin{frame}
\frametitle{Algunas fórmulas que contienen el número $\pi$} 

\begin{block}{Geometría}

\begin{itemize}

  \item Longitud de la circunferencia.
  \pause
  \item Área del círculo.
  \pause
  \item Área interior de la elipse.
  \pause
  \item Área del cono.
  \pause
  \item Área de la esfera.

\end{itemize}
\end{block}

\end{frame}

\subsection{Análisis}
\begin{frame}
\frametitle{Algunas fórmulas que contienen el número $\pi$}
\begin{block}{Análisis}
\begin{itemize}
  \item Fórmula de Leibniz.
  \pause
  \item Producto de Wallis.
  \pause
  \item Fórmula de Euler.
  \pause
  \item Fórmula de Stirling.
  \pause
  \item Método de Montecarlo

\end{itemize}
\end{block}

\end{frame}

\subsection{Cálculo} 
\begin{frame}

\begin{block}{Cálculo}
\begin{itemize}
  \item Área limitada por la astroide: $\frac{3}{8}\pi a^2 $.
  \pause

  \item Área de la región comprendida por el eje X y un arco de la cicloide: $3 \pi a^2.$

\end{itemize}
\end{block}

\end{frame}

\begin{frame}
\frametitle{Bibliografía}
\begin{thebibliography}
  \beamertermplatebookbibitems
  \bibitem[Internet]{wikipedia}
  {\small $es.wikipedia.org/wiki/Método\_de\_bisección\#Algoritmo$}
  
  \beamertermplatebookbibitems
  \bibitem[Internet]{Juegos de lógica}
  {\small $www.juegosdelogica.com/numero\_\pi.htm$}
  
\end{thebibliography}
\end{frame}

\end{document}
